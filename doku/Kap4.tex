% =============================================================================
% Kapitel 4: Deployment & Betrieb
% =============================================================================

\section{Deployment und Betrieb}

\subsection{Kompatibilität}

TechPulse benötigt \textbf{Python 3.11} oder höher, da es Dataclasses, Type Hints mit \texttt{Optional[T]} und dem
Walrus-Operator (\texttt{:=}) verwendet.

Die Anwendung läuft unter Linux, macOS und Windows. Alle externen Abhängigkeiten stehen in der \texttt{requirements.txt}:

\begin{table}[ht]
\centering
\begin{tabular}{@{}llp{6.5cm}@{}}
\toprule
\textbf{Bibliothek} & \textbf{Version} & \textbf{Zweck} \\
\midrule
feedparser  & $\geq$ 6.0.10 & RSS/Atom-Feed-Parser \\
pyyaml      & $\geq$ 6.0    & YAML-Parser für die Konfiguration \\
jinja2      & $\geq$ 3.1.2  & Template-Engine für HTML-Reports \\
\bottomrule
\end{tabular}
\caption{Externe Abhängigkeiten von TechPulse}
\label{tab:dependencies}
\end{table}

\subsection{Konfiguration}

Die gesamte Konfiguration ist in \texttt{config/config.yaml} abgebildet. Im Folgenden die einzelnen Abschnitte:

\subsubsection*{Feeds}

Der \texttt{feeds}-Abschnitt definiert die Quellen. Pro Feed werden vier Felder erwartet:

\begin{lstlisting}[style=yaml, caption={Feed-Konfiguration (Auszug aus config.yaml)}]
feeds:
  - name: "Heise Developer"
    url: "https://www.heise.de/developer/rss/news-atom.xml"
    category: "Tech News DE"
    priority: 1.2
\end{lstlisting}

\begin{itemize}
    \item \texttt{name}: Anzeigename im Report und in Log-Meldungen
    \item \texttt{url}: Feed-URL (muss mit \texttt{http://} oder
          \texttt{https://} beginnen)
    \item \texttt{category}: Gruppierung im Report
    \item \texttt{priority}: Quellenpriorität für das Scoring (1.0 = normal,
          $>$1 = höher)
\end{itemize}

\subsubsection*{Output}

\begin{lstlisting}[style=yaml, caption={Output-Konfiguration}]
output:
  directory: "output"
  filename_prefix: "techpulse_report"
  max_articles: 50
  min_score: 0.5
\end{lstlisting}

\begin{itemize}
    \item \texttt{directory}: Zielverzeichnis (wird automatisch erstellt)
    \item \texttt{filename\_prefix}: Präfix + Zeitstempel ergibt den Dateinamen
          (z.\,B. \texttt{techpulse\_report\_\allowbreak 20260214\_103015.html})
    \item \texttt{max\_articles}: Maximale Artikelanzahl (0 = unbegrenzt)
    \item \texttt{min\_score}: Artikel unter diesem Score werden ausgeschlossen
\end{itemize}

\subsubsection*{Interessen}

Der \texttt{interests}-Abschnitt definiert Themengebiete mit den Keywords und der Gewichtung:

\begin{lstlisting}[style=yaml, caption={Interessen-Konfiguration (Auszug aus config.yaml)}]
interests:
  - name: "Kuenstliche Intelligenz"
    keywords:
      - "AI"
      - "Machine Learning"
      - "GPT"
      - "LLM"
    weight: 2.0
\end{lstlisting}

Keywords werden case-insensitiv mit Word-Boundary-Matching geprüft. Das \texttt{weight}-Feld beeinflusst den
jeweiligen Scoring-Multiplikator.

\subsubsection*{Logging und Fetching}

Beide der folgende Abschnitte sind optional. Fehlende Werte werden durch Defaults ersetzt (\texttt{\_apply\_defaults()},
\texttt{src/config\_loader.py:75--112}).

\begin{lstlisting}[style=yaml, caption={Logging- und Fetching-Konfiguration}]
logging:
  level: "INFO"        # DEBUG, INFO, WARNING, ERROR, CRITICAL
  directory: "logs"
  filename: "aggregator.log"
  console: true

fetching:
  timeout: 10          # HTTP-Timeout pro Feed in Sekunden
  max_workers: 5       # Parallele Threads fuer Feed-Abruf
  user_agent: "TechPulse RSS Aggregator/1.0"
\end{lstlisting}

\newpage

\subsection{Installation und Ausführung}

\subsubsection*{Schritt 1: Virtuelle Umgebung einrichten}

\begin{lstlisting}[style=bash, caption={Virtuelle Umgebung erstellen und aktivieren}]
$ python3 -m venv .venv
$ source .venv/bin/activate      # Linux/macOS
$ .venv\Scripts\activate         # Windows
\end{lstlisting}

\subsubsection*{Schritt 2: Abhängigkeiten installieren}

\begin{lstlisting}[style=bash, caption={Abhängigkeiten installieren}]
$ pip install -r requirements.txt
\end{lstlisting}

\subsubsection*{Schritt 3: Ausführen}

\begin{lstlisting}[style=bash, caption={TechPulse ausführen}]
# Standard-Konfiguration verwenden:
$ python main.py

# Eigene Konfiguration und verbose Ausgabe:
$ python main.py --config my_config.yaml --verbose

# Testlauf ohne Report-Speicherung:
$ python main.py --dry-run

# Output-Verzeichnis ueberschreiben:
$ python main.py --output /tmp/reports
\end{lstlisting}

Nach der Ausführung des Skripts, gibt TechPulse eine Zusammenfassung auf der Konsole aus und gibt den Pfad zum generierten
HTML-Report aus.

\subsection{Produktiver Einsatz}

Ich lasse TechPulse auf einem Raspberry Pi laufen. Ein Cron-Job startet das Skript täglich morgens und legt den Report in
einem Verzeichnis ab, das ich über den Browser abrufen kann.

\begin{lstlisting}[style=bash, caption={Cron-Job für tägliche Ausführung}]
# Crontab-Eintrag: taeglich um 7:00 Uhr
0 7 * * * cd /home/pi/techPulse && \
  .venv/bin/python main.py >> logs/cron.log 2>&1
\end{lstlisting}

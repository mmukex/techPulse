% =============================================================================
% TechPulse - Schriftliche Ausarbeitung
% FH Südwestfalen | Modul: Skriptsprachen
% =============================================================================
\documentclass[a4paper,12pt]{article}

% --- Encoding & Sprache ---
\usepackage[ngerman]{babel}

% --- Layout ---
\usepackage[
    left=2.5cm,
    right=2.5cm,
    top=2.5cm,
    bottom=2.5cm
]{geometry}
\usepackage{parskip}
\usepackage{fancyhdr}
\setlength{\headheight}{14.5pt}
\usepackage{tocloft}

% --- Grafiken & Links ---
\usepackage{graphicx}
\usepackage[
    colorlinks=true,
    linkcolor=black,
    urlcolor=blue,
    citecolor=black
]{hyperref}

% --- Mathe ---
\usepackage{amsmath}

% --- Code-Listings ---
\usepackage{xcolor}
\usepackage{listings}

\definecolor{codegreen}{rgb}{0,0.5,0}
\definecolor{codegray}{rgb}{0.5,0.5,0.5}
\definecolor{codepurple}{rgb}{0.58,0,0.82}
\definecolor{backcolour}{rgb}{0.95,0.95,0.95}

\lstset{
    language=Python,
    backgroundcolor=\color{backcolour},
    basicstyle=\ttfamily\small,
    keywordstyle=\color{blue}\bfseries,
    stringstyle=\color{codegreen},
    commentstyle=\color{codegray}\itshape,
    numberstyle=\tiny\color{codegray},
    numbers=left,
    numbersep=8pt,
    breaklines=true,
    breakatwhitespace=false,
    showstringspaces=false,
    frame=single,
    framerule=0.5pt,
    rulecolor=\color{codegray},
    tabsize=4,
    captionpos=b,
    xleftmargin=1.5em,
    framexleftmargin=1.5em,
    aboveskip=1em,
    belowskip=1em,
    literate={ä}{{\"a}}1 {ö}{{\"o}}1 {ü}{{\"u}}1
             {Ä}{{\"A}}1 {Ö}{{\"O}}1 {Ü}{{\"U}}1
             {ß}{{\ss}}1
}

% Zusätzlicher Stil für YAML-Listings
\lstdefinestyle{yaml}{
    language={},
    basicstyle=\ttfamily\small,
    keywordstyle=\color{blue},
    stringstyle=\color{codegreen},
    commentstyle=\color{codegray}\itshape,
    morecomment=[l]{\#},
    frame=single,
    backgroundcolor=\color{backcolour},
    numbers=left,
    numberstyle=\tiny\color{codegray},
    numbersep=8pt,
    xleftmargin=1.5em,
    framexleftmargin=1.5em,
}

% Stil für Bash/Terminal-Listings
\lstdefinestyle{bash}{
    language=bash,
    basicstyle=\ttfamily\small,
    keywordstyle=\color{blue},
    stringstyle=\color{codegreen},
    commentstyle=\color{codegray}\itshape,
    frame=single,
    backgroundcolor=\color{backcolour},
    numbers=none,
    xleftmargin=1.5em,
    framexleftmargin=1.5em,
}

% --- Makros ---
\newcommand{\name}{mmukex}
\newcommand{\datum}{\today}

% --- Kopf-/Fußzeilen ---
\pagestyle{fancy}
\fancyhf{}
\fancyhead[L]{\small TechPulse -- Modularbeit Skriptsprachen}
\fancyhead[R]{\small mmukex}
\fancyfoot[C]{\thepage}
\renewcommand{\headrulewidth}{0.4pt}
\renewcommand{\footrulewidth}{0pt}

% --- Tabellen ---
\usepackage{booktabs}
\usepackage{array}
\usepackage{longtable}

% =============================================================================
% Titelseite
% =============================================================================
\begin{document}
\pagenumbering{Alph}

\begin{titlepage}
    \centering
    \vspace*{2cm}

    {\Huge\bfseries TechPulse\par}
    \vspace{0.5cm}
    {\Large RSS Tech-News-Aggregator\par}
    \vspace{0.3cm}
    {\large mit Keyword-Filterung und Scoring-System\par}

    \vspace{2cm}

    {\large Schriftliche Ausarbeitung\par}
    \vspace{0.3cm}
    {\large im Modul \textbf{Skriptsprachen}\par}

    \vspace{2cm}

    \begin{tabular}{rl}
        \textbf{Autor:}          & Max Meier \\
        \textbf{Matrikelnummer:} & 30394407 \\
        \textbf{Hochschule:}     & FH Südwestfalen \\
        \textbf{Datum:}          & \today \\
    \end{tabular}

    \vfill
\end{titlepage}

% =============================================================================
% Inhaltsverzeichnis
% =============================================================================
\clearpage
\pagenumbering{arabic}
\tableofcontents
\newpage

% =============================================================================
% Kapitel
% =============================================================================
% =============================================================================
% Kapitel 1: Einleitung
% =============================================================================

\section{Einleitung}

\subsection{Der Name}

Ich wollte einen kurzen Namen, der sofort vermittelt, worum es geht. \textit{Tech} für Technologie, \textit{Pulse} für
den Puls, also am Puls der Tech-News bleiben. Der Name war nach kurzem Brainstorming klar und ist geblieben.

\subsection{Das Projekt}

Als Entwickler lese ich regelmäßig verschiedene Blogs. Das Problem dabei: relevante Artikel manuell zu finden kostet
Zeit und viele interessante Themen gehen unter. Ich wollte ein Tool, das mir diese Arbeit abnimmt und für mich persönlich
individualisierbar ist.

Beruflich bin ich als Entwickler mit Fokus auf nicht-funktionale Anforderungen zuständig. Dazu gehört, über Änderungen in verwendeten
Libraries und Frameworks auf dem Laufenden zu bleiben. Dies betrifft besonders Sicherheitslücken, Breaking Changes oder
neue Möglichkeiten, die den Entwicklungsalltag vereinfachen. Gerade Coding Agents wie z.B.
Claude Code\footnote{Claude Code: \url{https://docs.anthropic.com/en/docs/claude-code}} oder
GitHub Copilot\footnote{GitHub Copilot:\url{https://github.com/features/copilot}} entwickelen sich rasant und eröffnen
uns Entwickler neue Wege in der Modernisierung von Legacy-Code.

Aktuell nutzen wir solche Tools beispielsweise, um veraltete
Hibernate DetachedCriteria\footnote{Hibernate DetachedCriteria:\url{https://docs.jboss.org/hibernate/orm/6.6/javadocs/org/hibernate/criterion/DetachedCriteria.html}}
auf JPA Criteria Queries\footnote{JPA Criteria API:\url{https://jakarta.ee/specifications/persistence/3.1/jakarta-persistence-spec-3.1\#a6925}}
zu migrieren. Solche Entwicklungen frühzeitig mitzubekommen spart dem Team langfristig Aufwand und damit Ressourcen.
TechPulse ist aus diesem Bedarf heraus entstanden. Anstatt täglich mehrere Quellen manuell durchsuchen zu müssen, trägt
TechPulse automatisiert Tech-News zusammen.

TechPulse ist ein RSS-News-Aggregator, der mehrere Feeds parallel abruft, nach konfigurierbaren Keywords filtert,
ein persönliches Relevanz-Scoring berechnet und die Ergebnisse als HTML-Report aufbereitet. Konkret passiert dabei Folgendes:

\begin{enumerate}
    \item \textbf{Paralleler Feed-Abruf:} Mehrere RSS-\footnote{RSS 2.0 Spezifikation:
    \url{https://www.rssboard.org/rss-specification}} und Atom-Feeds\footnote{Atom Syndication Format (RFC~4287):
    \url{https://www.rfc-editor.org/rfc/rfc4287}} werden gleichzeitig über einen Thread-Pool abgerufen.
    \item \textbf{Keyword-Filterung:} Artikel werden anhand konfigurierter Keywords gefiltert.
    \item \textbf{Relevanz-Scoring:} Jeder Artikel bekommt einen Score basierend auf Anzahl und Position der Keyword-Matches
    sowie der individuell konfigurierten Gewichtung des Interessengebiets.
    \item \textbf{HTML-Report:} Die bewerteten Artikel werden in einem HTML-Report mit einer Score-Angabe, Gruppierungen und
    Statistiken ausgegeben.
    \item \textbf{CLI-Interface:} Über die Kommandozeile lässt sich der gesamte Ablauf steuern.
\end{enumerate}

Mir persönlich war dabei wichtig, auf saubere Modultrennung zu achten. Außerdem wollte ich die
SOLID-Prinzipien\footnote{Robert C. Martin:\textit{Design Principles and Design Patterns}, 2000. Übersicht:\url{https://de.wikipedia.org/wiki/Prinzipien_objektorientierten_Designs}}
umzusetzen. Jedes Modul hat genau eine Aufgabe. Die Konfiguration ist komplett über eine externe YAML-Datei
individualisierbar. Anpassungen sind also ohne Code-Änderungen möglich.

Technisch setzt TechPulse auf
\texttt{feedparser}\footnote{feedparser -- Universal Feed Parser:\url{https://github.com/kurtmckee/feedparser}} für das Feed-Parsing,
\texttt{Jinja2}\footnote{Jinja2 Template Engine:\url{https://jinja.palletsprojects.com/}} für die Report-Generierung und
\texttt{PyYAML}\footnote{PyYAML -- YAML Parser for Python: \url{https://pyyaml.org/}} für das Management der Konfigurationen.

\newpage
% =============================================================================
% Kapitel 2: System-Architektur
% =============================================================================

\section{System-Architektur}

\subsection{Überblick und Technologie-Stack}

TechPulse ist in Python geschrieben und nutzt für jeden Verarbeitungsschritt eine Bibliothek. Die
Tabelle~\ref{tab:techstack} zeigt den Tech-Stack im Überblick.

\begin{table}[ht]
\centering
\begin{tabular}{@{}llp{7cm}@{}}
\toprule
\textbf{Komponente} & \textbf{Technologie} & \textbf{Beschreibung} \\
\midrule
Backend         & Python 3.11+     & Für Dataclasses, Type Hints und Threading \\
Feed-Parsing    & feedparser 6.0+  & Universal-Parser für RSS/Atom-Feeds \\
Konfiguration   & PyYAML 6.0+      & YAML-Parser mit Validierung \\
Templates       & Jinja2 3.1+      & Template-Engine mit Auto-Escaping \\
Report          & HTML/CSS         & Statische Reports mit eingebettetem CSS \\
CLI             & argparse         & Standardbibliothek für Kommandozeilen-Argumente \\
Logging         & logging          & Standardbibliothek mit File-/Console-Handler \\
\bottomrule
\end{tabular}
\caption{Technologie-Stack von TechPulse}
\label{tab:techstack}
\end{table}

\subsection{Systemkomponenten}

Das Projekt besteht aus sieben Modulen. Jedes Modul hat genau eine Aufgabe. Dies ist folgend in der Tabelle~\ref{tab:modules}
beschrieben:

\begin{table}[ht]
\centering
\begin{tabular}{@{}llp{6.5cm}@{}}
\toprule
\textbf{Modul} & \textbf{Zeilen} & \textbf{Aufgabe} \\
\midrule
\texttt{main.py}               & 259 & CLI-Orchestrierung: Argumente parsen,
                                        Workflow steuern, Statistiken ausgeben \\
\texttt{src/config\_loader.py} & 218 & YAML laden, Defaults setzen, validieren \\
\texttt{src/feed\_parser.py}   & 291 & Feeds parallel abrufen und in
                                        Article-Objekte parsen \\
\texttt{src/filter.py}         & 214 & Keyword-Filterung mit
                                        Word-Boundary-Matching \\
\texttt{src/scorer.py}         & 250 & Gewichtete Relevanz-Scores berechnen \\
\texttt{src/output\_generator.py} & 262 & HTML-Reports per Jinja2 generieren \\
\texttt{src/logger.py}         &  90 & File- und Console-Logging konfigurieren \\
\bottomrule
\end{tabular}
\caption{Modulübersicht von TechPulse}
\label{tab:modules}
\end{table}

\subsection{Projektstruktur}

Die Verzeichnisstruktur von TechPulse lässt sich so abbilden:

\begin{lstlisting}[style=bash, caption={Projektstruktur}]
techPulse/
  config/
    config.yaml           # Zentrale Konfiguration
  src/
    __init__.py
    config_loader.py      # YAML laden + validieren
    feed_parser.py        # Feeds abrufen + parsen
    filter.py             # Keyword-Filterung
    scorer.py             # Relevanz-Scoring
    output_generator.py   # HTML-Report generieren
    logger.py             # Logging-Setup
  templates/
    report_template.html  # Jinja2 HTML-Template
  logs/
    aggregator.log
  output/
    *.html                # Generierte Reports
  main.py                 # CLI-Einstiegspunkt
  requirements.txt
  pyproject.toml
\end{lstlisting}

\subsection{SOLID-Prinzipien}

Ich habe beim Entwurf darauf geachtet die SOLID-Prinzipien bewusst umzusetzen. Im Folgenden beschreibe ich, wie
sich das konkret im Code zeigt.

\subsubsection*{Single Responsibility Principle (SRP)}

Jedes Modul hat genau eine Aufgabe: \texttt{feed\_parser.py} ruft Feeds ab, \texttt{filter.py} filtert,
\texttt{scorer.py} berechnet Scores, \texttt{output\_generator.py} generiert Reports. Das gilt auch innerhalb der
einzelnen Module. So ist beispielsweise die Funktion \texttt{\_build\_word\_boundary\_pattern()} in \texttt{src/filter.py:26--31}
nur für die Regex-Pattern-Konstruktion zuständig.

\subsubsection*{Open/Closed Principle (OCP)}

Neue Feeds, Interessen und Keywords werden ausschließlich über die \texttt{config/config.yaml} hinzugefügt. Der Code
muss dafür nicht angepasst werden. Auch die Gewichtung einzelner Interessen ist über den \texttt{weight}-Parameter
steuerbar.

\subsubsection*{Dependency Inversion Principle (DIP)}

Die Module kommunizieren nicht direkt über die Konfigurationsquelle, sondern erhalten ihre Parameter als Dictionaries
von der Funktion \texttt{config\_loader.py}.
Kein Modul liest selbst die YAML-Konfiguration oder kennt das Dateiformat der Konfiguration. Streng genommen
ist das noch kein vollständiges DIP, da die Module einander direkt importieren
(z.,B.\ importiert \texttt{scorer.py} die Funktion \texttt{keyword\_matches} aus \texttt{filter.py}). Um das Prinzip von
YAGNI\footnote{YAGNI-Prinzip: \url{https://de.wikipedia.org/wiki/YAGNI}} nicht zu brechen, wurde auf eine Umsetzung über
Abstraktion verzichtet.

Das Liskovsches Substitutionsprinzip verlangt, dass Subtypen ihren Basistyp ersetzen können, ohne das Programmverhalten
zu ändern. TechPulse nutzt keine Vererbung. Das Interface-Segregation-Prinzip besagt, dass Clients nicht von Methoden
abhängen sollen, die sie nicht nutzen. Das setzt formale Interfaces voraus, die aufgrund von YAGNI nicht implementiert
worden sind.


\subsection{Datenfluss}

Die Verarbeitung läuft als lineare Pipeline. Jedes Modul nimmt die Daten des vorherigen Schritts entgegen, verarbeitet
sie und reicht sie weiter. Ich habe mich bewusst für diesen Aufbau entschieden, weil es die Module voneinander
entkoppelt. Jeder Schritt hat einen klar definierten Input und Output.

Zwischen den Modulen dienen \texttt{Article}-Dataclasses als der einheitliche Datentyp. Dieser Typ ist angereichert mit
Listen gefundener Keywords und berechneten Scores als Python-Tupel. Dadurch arbeiten alle Module mit derselben Datenstruktur,
ohne voneinander zu wissen.

Abbildung~\ref{fig:activity} zeigt den vollständigen Datenfluss als UML-Aktivitätsdiagramm mit Entscheidungsknoten für
die Fehlerbehandlung und der Score-Filterung. Die Details der einzelnen Schritte beschreibt das
Kapitel~\ref{sec:implementierung}.

\begin{figure}[htbp]
    \centering
    \includegraphics[width=0.7\textwidth,height=0.85\textheight,keepaspectratio]{aktivitaetendiagramm}
    \caption{Aktivitätsdiagramm: Datenfluss der TechPulse-Pipeline}
    \label{fig:activity}
\end{figure}
\newpage
% =============================================================================
% Kapitel 3: Implementierung
% =============================================================================

\section{Implementierung}
\label{sec:implementierung}

\subsection{Backend-Implementierung}

Die in Kapitel~\ref{sec:architektur} beschriebene Modulstruktur wird im Folgenden auf Implementierungsebene betrachtet.
Die einzelnen Abschnitte sollen die zentralen Designentscheidungen, verwendeten Algorithmen und relevante Codeausschnitte
der fünf Kernmodule erläutern.

\subsubsection{Konfigurationsmanagement}

Welche Effekte die Konfigruation innerhalb der Anwendung hat, ist in
\texttt{src/config\_loader.py} implementiert. Alle Parameter wie Feed-URLs, Keywords, Logging-Einstellungen werden zentral
in \texttt{config/config.yaml}\footnote{YAML Spezifikation: \url{https://yaml.org/spec/}} beschrieben.

Der Ladeprozess für die Konfiguration besteht aus drei Schritten:

\begin{enumerate}
    \item \textbf{YAML laden:} \texttt{\_load\_yaml\_file()} in
          \texttt{src/config\_loader.py:52--71} liest die Datei mit
          \texttt{yaml.safe\_load()}\footnote{\texttt{safe\_load()} statt
          \texttt{yaml.load()} verhindert Code-Injection über YAML-Tags.} ein.
    \item \textbf{Defaults setzen:} \texttt{\_apply\_defaults()} (Zeile~74--111) ergänzt fehlende optionale Abschnitte mit
            Standardwerten. Durch
          \texttt{setdefault()} bleiben vorhandene Werte erhalten.
    \item \textbf{Validieren:} \texttt{validate\_config()} (Zeile~114--120) delegiert an spezialisierte
            Validierungsfunktionen pro Abschnitt.
\end{enumerate}

Die Validierung prüft, ob alle Feeds die Attribute \texttt{name}, \texttt{url} und \texttt{category} enthalten
(\texttt{\_validate\_feeds()}, Zeile~123--150) und ob die Interessen gültige Keywords bzw. eine positive Gewichtungen haben
(\texttt{\_validate\_interests()}, Zeile~152--186). Bei Fehlern wirft der Loader ein \texttt{ConfigurationError} mit dem
fehlerhaften Konfigurationsschlüssel:

\begin{lstlisting}[caption={ConfigurationError mit Schlüssel-Referenz (config\_loader.py:15--20)}]
class ConfigurationError(Exception):
    def __init__(self, message: str,
                 config_key: Optional[str] = None):
        self.config_key = config_key
        super().__init__(message)
\end{lstlisting}

\newpage
Die Konfigurationsdatei gliedert sich in fünf Abschnitte:

\begin{table}[ht]
\centering
\begin{tabular}{@{}llp{7cm}@{}}
\toprule
\textbf{Abschnitt} & \textbf{Pflicht} & \textbf{Inhalt} \\
\midrule
\texttt{feeds}     & Ja  & RSS/Atom-Feeds mit Name, URL, Kategorie,
                           Priorität \\
\texttt{interests} & Ja  & Interessengebiete mit Keywords und Gewichtung \\
\texttt{output}    & Nein & Report-Verzeichnis, Dateiname-Präfix,
                            Max-Artikel, Min-Score \\
\texttt{logging}   & Nein & Log-Level, Verzeichnis, Dateiname, Format \\
\texttt{fetching}  & Nein & HTTP-Timeout, Max-Worker, User-Agent \\
\bottomrule
\end{tabular}
\caption{Konfigurationsabschnitte der config.yaml}
\label{tab:config_sections}
\end{table}
\subsubsection{Feed-Parser}

Die Funktion \texttt{src/feed\_parser.py} ruft die Feeds ab und parst diese. Die zentrale Datenstruktur ist die
\texttt{Article}-Dataclass\footnote{Python Dataclasses: \url{https://docs.python.org/3/library/dataclasses.html}}:

\begin{lstlisting}[caption={Article-Dataclass (feed\_parser.py:23--51)}]
@dataclass
class Article:
    title: str
    link: str
    description: str = ""
    published: Optional[datetime] = None
    feed_name: str = ""
    category: str = ""
    author: str = ""
    source_priority: float = 1.0

    def __post_init__(self):
        self.title = self.title.strip() if self.title else ""
        self.description = (
            self.description.strip()
            if self.description else ""
        )
\end{lstlisting}

Die Whitespaces aus Titel und Beschreibung werden automatisch mit der Funktion \texttt{\_\_post\_init\_\_} bereinigt. Dies ist
notwendig, da die Feed-Daten oft führende oder folgende Leerzeichen enthalten.

Der parallele Abruf läuft über \texttt{fetch\_all\_feeds()} in Zeile~218--290. Die Funktion nutzt einen
\texttt{ThreadPoolExecutor}\footnote{Python concurrent.futures:\url{https://docs.python.org/3/library/concurrent.futures.html}}
mit einer konfigurierbarer Worker-Anzahl. Tests haben mit fünf Workern gute Ergebnisse erzielt:

\begin{lstlisting}[caption={Paralleler Feed-Abruf (feed\_parser.py:243--259)}]
with ThreadPoolExecutor(max_workers=max_workers) as executor:
    future_to_feed = {}
    for feed_config in feed_configs:
        future = executor.submit(
            fetch_feed,
            url=feed_config['url'],
            name=feed_config['name'],
            category=feed_config['category'],
            priority=feed_config.get('priority', 1.0),
            timeout=timeout,
            user_agent=user_agent
        )
        future_to_feed[future] = feed_config['name']
\end{lstlisting}

\texttt{as\_completed()} liefert die Futures in der Reihenfolge wie sie terminiert wurden. Damit werden schnelle Feeds
also zuerst verarbeitet. Die Artikel werden nach dem neusten Veröffentlichungsdatum sortiert, wobei \texttt{datetime.min}
als ein Fallback für die Artikel ohne Datum dient.

Innerhalb von \texttt{fetch\_feed()} (Zeile~54--111) mappt \texttt{\_parse\_feed\_entry()} (Zeile~114--149)
jeden Feed-Eintrag in ein \texttt{Article}-Objekt. Dabei werden verschiedene Feed-Formate normalisiert:
\texttt{\_extract\_description()} prüft die Felder \texttt{summary},
\texttt{description} und \texttt{content},da RSS~2.0, Atom und andere Formate unterschiedliche Feldnamen verwenden.

\subsubsection{Keyword-Filter}

\texttt{src/filter.py} filtert Artikel anhand der konfigurierten Keywords. Der
Kern ist ein Regex-basiertes Word-Boundary-Matching\footnote{Python - Regular Expressions: \url{https://docs.python.org/3/library/re.html}}
über die Funktion \texttt{\_build\_word\_boundary\_pattern()} (Zeile~26--31):

\begin{lstlisting}[caption={Word-Boundary-Pattern (filter.py:23--31)}]
TITLE_MATCH_WEIGHT = 2

def _build_word_boundary_pattern(keyword: str) -> str:
    return r'\b' + re.escape(keyword.lower()) + r'\b'
\end{lstlisting}

Die Word-Boundaries (\texttt{\textbackslash b}) stellen sicher, dass nur ganze
Wörter matchen -- \glqq AI\grqq{} trifft also nicht in \glqq MAIL\grqq{} oder
\glqq FAIR\grqq{}. \texttt{re.escape()} schützt vor Sonderzeichen in Keywords.

\texttt{keyword\_matches()} (Zeile~34--61) nutzt ein \texttt{Set} zur Duplikatprüfung, damit jedes Keyword pro Text maximal
einmal zählt:

\begin{lstlisting}[caption={Keyword-Matching mit Duplikatprüfung (filter.py:48--61)}]
text_lower = text.lower()
found_keywords: List[str] = []
seen: Set[str] = set()

for keyword in keywords:
    if keyword in seen:
        continue
    pattern = _build_word_boundary_pattern(keyword)
    if re.search(pattern, text_lower):
        found_keywords.append(keyword)
        seen.add(keyword)
return found_keywords
\end{lstlisting}

Die Funktion \texttt{filter\_articles()} (Zeile~93--134) iteriert über alle Artikel und findet für jeden das am besten passende
Interesse. Die Zuordnung übernimmt die Funktion \texttt{\_find\_best\_interest\_match()} (Zeile~137--174), die die Titel- und
Beschreibungs-Matches getrennt bewertet. Die Titel-Matches werden mit dem Faktor \texttt{TITLE\_MATCH\_WEIGHT = 2} gewichtet,
da ein Keyword im Titel der Erfahrung nach ein stärkerer Relevanzindikator ist.

Für Statistiken bietet \texttt{get\_keyword\_statistics()} (Zeile~204--213) eine Häufigkeitsanalyse über \texttt{collections.Counter}:

\begin{lstlisting}[caption={Keyword-Statistik mit Counter (filter.py:208--213)}]
keyword_counts: Counter[str] = Counter()
for _, keywords, _ in filtered_articles:
    keyword_counts.update(keywords)
return dict(keyword_counts.most_common())
\end{lstlisting}

\subsubsection{Scoring-System}

Für jeden gefilterten Artikel berechnet die Funktion \texttt{src/scorer.py} einen Relevanz-Score. Vier Faktoren fließen
dort ein:

\begin{itemize}
    \item Anzahl der Keyword-Matches in der Beschreibung
    \item Anzahl der Keyword-Matches im Titel (höher gewichtet)
    \item Gewichtung des Interessengebiets (\texttt{weight})
    \item Quellenpriorität des Feeds (\texttt{source\_priority})
\end{itemize}

Die Formel dahinter:

\begin{equation}
\text{Score} = \left(\text{desc\_matches} \times w + \text{title\_matches}
\times 1{,}5 \times w\right) \times \text{source\_priority}
\label{eq:scoring}
\end{equation}

wobei $w$ die konfigurierte Gewichtung des Interesses ist. Die zentrale
Implementierung ist in \texttt{\_compute\_score\_breakdown()}
(\texttt{src/scorer.py:36--64}) implementiert:

\begin{lstlisting}[caption={Zentrale Scoring-Logik (scorer.py:36--64)}]
TITLE_MULTIPLIER = 1.5
BASE_SCORE_PER_MATCH = 1.0

def _compute_score_breakdown(
    article: Article,
    keywords: List[str],
    weight: float
) -> Dict[str, Any]:
    title_matches = keyword_matches(article.title, keywords)
    description_matches = keyword_matches(
        article.description, keywords
    )
    title_score = (len(title_matches) * TITLE_MULTIPLIER
                   * weight * BASE_SCORE_PER_MATCH)
    description_score = (len(description_matches)
                         * weight * BASE_SCORE_PER_MATCH)
    base_score = title_score + description_score
    total_score = base_score * article.source_priority
    return { ... }
\end{lstlisting}

Diese Funktion wird von den Funktionen \texttt{calculate\_score()} (Zeile~67--100) und
\texttt{calculate\_detailed\_score()} (Zeile~103--120) genutzt. Dadurch läuft das Keyword-Matching pro Aufruf nur einmal
statt dreimal (DRY-Prinzip).

Die Scores werden in drei Level eingeteilt:

\begin{table}[ht]
\centering
\begin{tabular}{@{}lcl@{}}
\toprule
\textbf{Level} & \textbf{Score-Bereich} & \textbf{Farbe im Report} \\
\midrule
Low    & $< 3{,}0$        & Gelb \\
Medium & $3{,}0$ -- $5{,}9$ & Blau \\
High   & $\geq 6{,}0$     & Grün \\
\bottomrule
\end{tabular}
\caption{Score-Level und ihre Schwellwerte (scorer.py:29--30)}
\label{tab:scorelevels}
\end{table}

Die Schwellwerte sind als Konstanten definiert:

\begin{itemize}
    \item \texttt{SCORE\_LEVEL\_LOW\_THRESHOLD = 3.0}
    \item \texttt{SCORE\_LEVEL\_HIGH\_THRESHOLD = 6.0}
\end{itemize}

\noindent Das \texttt{output\_generator.py}-Modul importiert diese Konstanten direkt.

\texttt{score\_all\_articles()} (Zeile~123--171) berechnet den Score über alle Interessen: ein Artikel mit
\glqq AI\grqq{}- und \glqq Security\grqq{}-Keywords erhält Punkte aus beiden Gebieten. Die Ergebnisse werden anschließend absteigend
sortiert.

\textbf{Beispielrechnung:} Ein Artikel mit \glqq AI\grqq{} im Titel, Interesse\glqq Künstliche Intelligenz\grqq{} (weight = 2.0), source\_priority = 1.2:

\[
\text{Score} = (0 \times 2{,}0 + 1 \times 1{,}5 \times 2{,}0) \times 1{,}2
= 3{,}0 \times 1{,}2 = 3{,}6 \quad (\text{Medium})
\]

\subsubsection{Report-Generator}

Die Funktion \texttt{src/output\_generator.py} generiert einen HTML-Report. Die Datenaufbereitung und das Rendering sind
dabei getrennt.

\texttt{prepare\_template\_data()} (Zeile~25--69) durchläuft drei Schritte:

\begin{enumerate}
    \item \textbf{Konvertierung:} \texttt{\_article\_to\_dict()} (Zeile~72--90)
          kovertiert Article-Objekte in Dictionaries, ergänzt um den \texttt{score},
          die \texttt{interest} und das \texttt{score\_level}.
    \item \textbf{Gruppierung:} \texttt{\_group\_articles\_by()} (Zeile~103--115)
          gruppiert Artikel nach beliebigen Schlüsseln über
          \texttt{collections.defaultdict}.
    \item \textbf{Statistiken:} \texttt{\_calculate\_statistics()}
          (Zeile~118--142) berechnet Durchschnitts-Score, Min/Max und Anzahl
          der Kategorien.
\end{enumerate}

\begin{lstlisting}[caption={Generische Artikel-Gruppierung (output\_generator.py:103--115)}]
def _group_articles_by(
    articles: List[Dict[str, Any]],
    key: str,
    default: str = "Sonstige"
) -> Dict[str, List[Dict]]:
    groups: Dict[str, List[Dict]] = defaultdict(list)
    for article in articles:
        group_name = article.get(key, default)
        groups[group_name].append(article)
    return dict(groups)
\end{lstlisting}

\texttt{generate\_html\_report()} (Zeile~145--190) konfiguriert die Jinja2-Environment und registriert Custom-Filter für
die Datums- und Score-Formatierung.

Der Report wird in \texttt{save\_report()} (Zeile~208--243) mit Zeitstempel im Dateinamen gespeichert.

Das HTML-Template liegt in \texttt{templates/report\_template.html} und wurde mit KI-Unterstützung erstellt
(ebenso wie die \texttt{README.md}). Das Template enthält eingebettetes CSS für ein Standalone-Dokument.

\begin{figure}[htbp]
    \centering
    \includegraphics[width=0.8\textwidth,height=0.8\textheight,keepaspectratio]{html_report}
    \caption{Beispiel eines generierten HTML-Reports}
    \label{fig:report_screenshot}
\end{figure}

\subsection{Logging}

\texttt{src/logger.py} richtet einen zentralen Logger namens \texttt{techpulse} ein, den alle Module über
\texttt{logging.getLogger("techpulse")}\footnote{Python logging: \url{https://docs.python.org/3/library/logging.html}} referenzieren.

\texttt{setup\_logger()} (Zeile~11--57) konfiguriert zwei Handler:

\begin{itemize}
    \item \textbf{FileHandler:} Schreibt Log-Meldungen in eine Datei
          (Standard: \texttt{logs/aggregator.log}), UTF-8-kodiert.
    \item \textbf{StreamHandler:} Gibt Log-Meldungen optional auf der Konsole
          aus (steuerbar über \texttt{logging.console} in der Konfiguration).
\end{itemize}

Um Loghandler-Duplikate bei mehrfachem Aufruf zu vermeiden, werden bestehende LohHandler vorher entfernt
(\texttt{logger.handlers.clear()}, Zeile~41).

\subsection{CLI-Interface}

Das Kommandozeilen-Interface ist in \texttt{parse\_arguments()} in
\texttt{main.py:27--65} mit der Bibliothek
\texttt{argparse}\footnote{Python argparse:
\url{https://docs.python.org/3/library/argparse.html}} implementiert:

\begin{table}[ht]
\centering
\begin{tabular}{@{}lll@{}}
\toprule
\textbf{Argument} & \textbf{Kurzform} & \textbf{Beschreibung} \\
\midrule
\texttt{-{}-config}   & \texttt{-c} & Pfad zur YAML-Konfigurationsdatei \\
\texttt{-{}-verbose}  & \texttt{-v} & Aktiviert DEBUG-Level Logging \\
\texttt{-{}-output}   & \texttt{-o} & Überschreibt das Output-Verzeichnis \\
\texttt{-{}-dry-run}  &             & Alle Schritte ohne Speichern ausführen \\
\bottomrule
\end{tabular}
\caption{CLI-Argumente von TechPulse}
\label{tab:cli_args}
\end{table}

Die \texttt{main()} in \texttt{main.py:103--254} orchestriert den Workflow in fünf nummerierten Schritten mit Fortschrittsanzeige.
CLI-Flags haben Vorrang vor der YAML-Konfiguration. Bei dem Argument \texttt{-{}-verbose} wird das Log-Level unabhängig
von der Config auf \texttt{DEBUG} gesetzt (Zeile~130--131).

Am Ende gibt \texttt{print\_statistics()} (Zeile~85--100) eine Zusammenfassung  aus. Die drei Artikel mit dem höchsten Score
werden als Vorschau angezeigt (Zeile~235--246).

\newpage
% =============================================================================
% Kapitel 4: Deployment & Betrieb
% =============================================================================

\section{Deployment und Betrieb}

\subsection{Kompatibilität}

TechPulse benötigt \textbf{Python 3.11} oder höher, da es Dataclasses, Type Hints mit \texttt{Optional[T]} und dem
Walrus-Operator (\texttt{:=}) verwendet.

Die Anwendung läuft unter Linux, macOS und Windows. Alle externen Abhängigkeiten stehen in der \texttt{requirements.txt}:

\begin{table}[ht]
\centering
\begin{tabular}{@{}llp{6.5cm}@{}}
\toprule
\textbf{Bibliothek} & \textbf{Version} & \textbf{Zweck} \\
\midrule
feedparser  & $\geq$ 6.0.10 & RSS/Atom-Feed-Parser \\
pyyaml      & $\geq$ 6.0    & YAML-Parser für die Konfiguration \\
jinja2      & $\geq$ 3.1.2  & Template-Engine für HTML-Reports \\
\bottomrule
\end{tabular}
\caption{Externe Abhängigkeiten von TechPulse}
\label{tab:dependencies}
\end{table}

\subsection{Konfiguration}

Die gesamte Konfiguration ist in \texttt{config/config.yaml} abgebildet. Im Folgenden die einzelnen Abschnitte:

\subsubsection*{Feeds}

Der \texttt{feeds}-Abschnitt definiert die Quellen. Pro Feed werden vier Felder erwartet:

\begin{lstlisting}[style=yaml, caption={Feed-Konfiguration (Auszug aus config.yaml)}]
feeds:
  - name: "Heise Developer"
    url: "https://www.heise.de/developer/rss/news-atom.xml"
    category: "Tech News DE"
    priority: 1.2
\end{lstlisting}

\begin{itemize}
    \item \texttt{name}: Anzeigename im Report und in Log-Meldungen
    \item \texttt{url}: Feed-URL (muss mit \texttt{http://} oder
          \texttt{https://} beginnen)
    \item \texttt{category}: Gruppierung im Report
    \item \texttt{priority}: Quellenpriorität für das Scoring (1.0 = normal,
          $>$1 = höher)
\end{itemize}

\subsubsection*{Output}

\begin{lstlisting}[style=yaml, caption={Output-Konfiguration}]
output:
  directory: "output"
  filename_prefix: "techpulse_report"
  max_articles: 50
  min_score: 0.5
\end{lstlisting}

\begin{itemize}
    \item \texttt{directory}: Zielverzeichnis (wird automatisch erstellt)
    \item \texttt{filename\_prefix}: Präfix + Zeitstempel ergibt den Dateinamen
          (z.\,B. \texttt{techpulse\_report\_\allowbreak 20260214\_103015.html})
    \item \texttt{max\_articles}: Maximale Artikelanzahl (0 = unbegrenzt)
    \item \texttt{min\_score}: Artikel unter diesem Score werden ausgeschlossen
\end{itemize}

\subsubsection*{Interessen}

Der \texttt{interests}-Abschnitt definiert Themengebiete mit den Keywords und der Gewichtung:

\begin{lstlisting}[style=yaml, caption={Interessen-Konfiguration (Auszug aus config.yaml)}]
interests:
  - name: "Kuenstliche Intelligenz"
    keywords:
      - "AI"
      - "Machine Learning"
      - "GPT"
      - "LLM"
    weight: 2.0
\end{lstlisting}

Keywords werden case-insensitiv mit Word-Boundary-Matching geprüft. Das \texttt{weight}-Feld beeinflusst den
jeweiligen Scoring-Multiplikator.

\subsubsection*{Logging und Fetching}

Beide der folgende Abschnitte sind optional. Fehlende Werte werden durch Defaults ersetzt (\texttt{\_apply\_defaults()},
\texttt{src/config\_loader.py:75--112}).

\begin{lstlisting}[style=yaml, caption={Logging- und Fetching-Konfiguration}]
logging:
  level: "INFO"        # DEBUG, INFO, WARNING, ERROR, CRITICAL
  directory: "logs"
  filename: "aggregator.log"
  console: true

fetching:
  timeout: 10          # HTTP-Timeout pro Feed in Sekunden
  max_workers: 5       # Parallele Threads fuer Feed-Abruf
  user_agent: "TechPulse RSS Aggregator/1.0"
\end{lstlisting}

\newpage

\subsection{Installation und Ausführung}

\subsubsection*{Schritt 1: Virtuelle Umgebung einrichten}

\begin{lstlisting}[style=bash, caption={Virtuelle Umgebung erstellen und aktivieren}]
$ python3 -m venv .venv
$ source .venv/bin/activate      # Linux/macOS
$ .venv\Scripts\activate         # Windows
\end{lstlisting}

\subsubsection*{Schritt 2: Abhängigkeiten installieren}

\begin{lstlisting}[style=bash, caption={Abhängigkeiten installieren}]
$ pip install -r requirements.txt
\end{lstlisting}

\subsubsection*{Schritt 3: Ausführen}

\begin{lstlisting}[style=bash, caption={TechPulse ausführen}]
# Standard-Konfiguration verwenden:
$ python main.py

# Eigene Konfiguration und verbose Ausgabe:
$ python main.py --config my_config.yaml --verbose

# Testlauf ohne Report-Speicherung:
$ python main.py --dry-run

# Output-Verzeichnis ueberschreiben:
$ python main.py --output /tmp/reports
\end{lstlisting}

Nach der Ausführung des Skripts, gibt TechPulse eine Zusammenfassung auf der Konsole aus und gibt den Pfad zum generierten
HTML-Report aus.

\subsection{Produktiver Einsatz}

Ich lasse TechPulse auf einem Raspberry Pi laufen. Ein Cron-Job startet das Skript täglich morgens und legt den Report in
einem Verzeichnis ab, das ich über den Browser abrufen kann.

\begin{lstlisting}[style=bash, caption={Cron-Job für tägliche Ausführung}]
# Crontab-Eintrag: taeglich um 7:00 Uhr
0 7 * * * cd /home/pi/techPulse && \
  .venv/bin/python main.py >> logs/cron.log 2>&1
\end{lstlisting}

\newpage
% =============================================================================
% Kapitel 5: Fazit & Ausblick
% =============================================================================

\section{Fazit}

\subsection{Zusammenfassung}

TechPulse ist im Sinne seiner geplanten Implementierung vollständig. Der
Aggregator übernimmt den gesamten Ablauf von der Nachrichtenbeschaffung bis
zum fertigen HTML-Report.

Das Projekt besteht aus sieben Modulen, die jeweils eine Aufgabe haben. Über
die YAML-Konfiguration können neue Feeds und Interessen ohne Code-Änderungen
hinzugefügt werden.

\subsection{Erreichte Ziele}

\begin{itemize}
    \item \textbf{Multi-Feed-Aggregation:} Paralleler Abruf mehrerer RSS/Atom-Feeds
          mit Fehlertoleranz pro Feed.
    \item \textbf{Keyword-Filterung:} Word-Boundary-Matching per Regex, keine
          Falsch-Positiven bei Teilwörtern.
    \item \textbf{Relevanz-Scoring:} Gewichtetes Scoring mit Titel-Bonus und
          Quellenpriorität.
    \item \textbf{HTML-Reports:} Jinja2-basierte Reports mit Score-Badges und
          XSS-Schutz.
    \item \textbf{Modulare Architektur:} Klare Modultrennung, Konfiguration extern
          über YAML.
    \item \textbf{CLI-Interface:} Konfiguration, Verbose-Modus und Dry-Run per
          Kommandozeile steuerbar.
\end{itemize}

\subsection{Lessons Learned}

Einige Erkenntnisse aus der Entwicklung, die ich hier festhalten möchte:

\begin{description}
    \item[feedparser-Kompatibilität:] \texttt{feedparser} liefert je nach
        Feed-Format unterschiedliche Feldnamen für denselben Inhalt. Beschreibungen
        stehen mal unter \texttt{summary}, mal unter \texttt{description} oder auch
        mal unter \texttt{content}. Ebenso variieren die Datumsfelder. Dies erfordert
        eine sorgfältige Normalisierung in der Parse-Schicht, was ich anfangs unterschätzt habe.

    \item[Jinja2 Template-Design:] Article-Objekte (Dataclasses) vor der Übergabe
        an Jinja2 in Dictionaries zu konvertieren hat sich als sinnvoll erwiesen.
        Templates arbeiten besser mit Dictionary-Zugriff als mit
        Attribut-Zugriff auf Objekte.

    \item[YAML-Validierung:] Frühzeitige Validierung der Konfiguration erspart
        kryptische Laufzeitfehler in späteren Schritten. Die Kombination aus
        Defaults für optionale Felder und strikter Validierung für Pflichtfelder
        hat sich für mich bewährt.

    \item[Word-Boundary-Matching:] Einfaches String-Matching führt bei kurzen
        Keywords wie \glqq AI\grqq{} oder \glqq Go\grqq{} zu zahlreichen
        False-Positives. Ein Regex mit \texttt{\textbackslash b} löst aber das Problem.
\end{description}


% =============================================================================
% Anhang
% =============================================================================
\newpage
% =============================================================================
% Anhang
% =============================================================================

\appendix
\section*{Anhang}
\addcontentsline{toc}{section}{Anhang}
\thispagestyle{empty}

\subsection*{Anhang A: Quellcode}

Link zum GitHub-Repository:\\
\url{https://github.com/mmukex/techPulse}

\bigskip

% =============================================================================
% Literaturverzeichnis
% =============================================================================

\newpage
\thispagestyle{empty}
\section*{Literaturverzeichnis}
\addcontentsline{toc}{section}{Literaturverzeichnis}

\begin{enumerate}
    \item Claude Code -- Anthropic\\
          \url{https://docs.anthropic.com/en/docs/claude-code}

    \item GitHub Copilot\\
          \url{https://github.com/features/copilot}

    \item Hibernate DetachedCriteria (ORM 6.6)\\
          \url{https://docs.jboss.org/hibernate/orm/6.6/javadocs/org/hibernate/criterion/DetachedCriteria.html}

    \item JPA Criteria API (Jakarta Persistence 3.1)\\
          \url{https://jakarta.ee/specifications/persistence/3.1/jakarta-persistence-spec-3.1\#a6925}

    \item RSS 2.0 Spezifikation\\
          \url{https://www.rssboard.org/rss-specification}

    \item Atom Syndication Format (RFC~4287)\\
          \url{https://www.rfc-editor.org/rfc/rfc4287}

    \item Robert C. Martin: \textit{Design Principles and Design Patterns}, 2000\\
          \url{https://de.wikipedia.org/wiki/Prinzipien_objektorientierten_Designs}

    \item feedparser -- Universal Feed Parser\\
          \url{https://github.com/kurtmckee/feedparser}

    \item Jinja2 Template Engine\\
          \url{https://jinja.palletsprojects.com/}

    \item PyYAML -- YAML Parser for Python\\
          \url{https://pyyaml.org/}

    \item YAGNI-Prinzip\\
          \url{https://de.wikipedia.org/wiki/YAGNI}

    \item YAML Spezifikation\\
          \url{https://yaml.org/spec/}

    \item Dokumentation zu Python 3.11+\\
          \url{https://docs.python.org/3/}
\end{enumerate}


% =============================================================================
% Erklaerung
% =============================================================================
\newpage
\thispagestyle{empty}
\section*{Eigenständigkeitserklärung}

Erklärung an Eides statt:

Hiermit erklären wir an Eides statt, dass wir die vorliegende Arbeit selbstständig verfasst haben, dass wir sie zuvor an keiner anderen Hochschule und in keinem anderen Studiengang als Prüfungsleistung eingereicht haben, und dass wir keine anderen als die angegebenen Quellen und Hilfsmittel genutzt haben.

Alle Stellen der Arbeit, die wörtlich oder sinngemäß aus Veröffentlichungen oder anderweitigen fremden Äußerungen entnommen wurden, sind als solche kenntlich gemacht.

\bigskip
\noindent
Dortmund, \datum

\vspace{2cm}

\noindent
Max Meier

Matrikelnummer: 30394407

\end{document}
